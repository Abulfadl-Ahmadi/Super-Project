\documentclass[11pt]{book}
\usepackage{dblfloatfix,enumitem,graphicx,adjustbox,fancyhdr,color,multicol,bm,amsthm,amsmath}
\usepackage{geometry}
\geometry{	%--------------تعیین حاشیه صفحه--------------
	a4paper,
	top=1.75cm,
	bottom=1cm,
	inner=1.25cm,
	outer=0.75cm	
}

\makeatletter
\newcommand{\mytitle}{\@title}
\makeatother
\pagestyle{fancy}

% \renewcommand{\chaptermark}[1]{\markboth{#1}{}}
\renewcommand{\sectionmark}[1]{\markboth{#1}{}}
\chead{}
\fancyhf{}
\fancyhead[CO,CE]{معادله درجه دوم}
\fancyhead[LO,RE]{\thepage}

\renewcommand{\adjboxvtop}{8mm}
%--------------تعیین فاصله بین تست ها--------------
\newcommand{\sol}{\bigskip \bigskip \bigskip \bigskip \bigskip \bigskip }

\newcommand{\riazi}[1]{\textbf{$\bm{#1}$}}
\newcommand{\quenstion}[1]{\item \adjustbox{valign=t}{\includegraphics[width=\linewidth]{Image/#1}}
	
	\sol
}

\newcommand{\quz}[1]{\item \adjustbox{valign=t}{\includegraphics[width=\linewidth]{Image/#1}}}

\usepackage{xepersian}	%تنظیمات فارسی نویسی
\settextfont{XB Niloofar}	%می باشد font فونت ها در داخل پوشه 
\setdigitfont{Yas}


\title{ جزوه ریاضیات و حسابان}
\author{ابوالفضل ا. | علیرضا ک.}

% -------------------------------------------------------------

%\newenvironment{mcquestions}{}{}
\newcommand{\test}[1]{\textbf{\item #1}}
\newsavebox{\answera}
\newsavebox{\answerb}
\newsavebox{\answerc}
\newsavebox{\answerd}
\newcommand{\answer}[4]{%
	\sbox\answera{ #1}%
	\sbox\answerb{ #2}%
	\sbox\answerc{ #3}%
	\sbox\answerd{ #4\sol}%
	\ifdim\wd\answera<0.165\textwidth%
	\ifdim\wd\answerb<0.165\textwidth%
	\ifdim\wd\answerc<0.165\textwidth%
	\ifdim\wd\answerd<0.165\textwidth%
	\begin{multicols}{4}
		\begin{itemize}
			\item[1)] \usebox{\answera}
			\item[2)] \usebox{\answerb}
			\item[3)] \usebox{\answerc}
			\item[4)] \usebox{\answerd}
		\end{itemize}
	\end{multicols}
	\else%
	\ifdim\wd\answera<0.4\textwidth%
	\ifdim\wd\answerb<0.4\textwidth%
	\ifdim\wd\answerc<0.4\textwidth%
	\ifdim\wd\answerd<0.4\textwidth%
	\begin{multicols}{2}
		\begin{itemize}
			\item[1)] \usebox{\answera}
			\item[3)] \usebox{\answerc}
			\item[2)] \usebox{\answerb}
			\item[4)] \usebox{\answerd}
		\end{itemize}
	\end{multicols}
	\else
	\begin{itemize} 
		\item[1)]  #1
		\item[2)]  #2
		\item[3)]  #3
		\item[4)]  #4
	\end{itemize}
	\fi
	\else
	\begin{itemize} 
		\item[1)] #1
		\item[2)] #2
		\item[3)] #3
		\item[4)] #4
	\end{itemize}
	\fi
	\else
	\begin{itemize} 
		\item[1)] #1
		\item[3)] #2
		\item[2)] #3
		\item[4)] #4
	\end{itemize}
	\fi
	\else
	\begin{itemize} 
		\item[1)] #1
		\item[3)] #2
		\item[2)] #3
		\item[4)] #4
	\end{itemize}
	\fi\fi
	\else
	\ifdim\wd\answera<0.4\textwidth%
	\ifdim\wd\answerb<0.4\textwidth%
	\ifdim\wd\answerc<0.4\textwidth%
	\ifdim\wd\answerd<0.4\textwidth%
	\begin{multicols}{2}
		\begin{itemize}
			\item[1)] \usebox{\answera}
			\item[3)] \usebox{\answerc}
			\item[2)] \usebox{\answerb}
			\item[4)] \usebox{\answerd}
		\end{itemize}
	\end{multicols}
	\else
	\begin{itemize} 
		\item[1)] #1
		\item[3)] #2
		\item[2)] #3
		\item[4)] #4
	\end{itemize}
	\fi
	\else
	\begin{itemize} 
		\item[1)] #1
		\item[3)] #2
		\item[2)] #3
		\item[4)] #4
	\end{itemize}
	\fi
	\else
	\begin{itemize} 
		\item[1)] #1
		\item[3)] #2
		\item[2)] #3
		\item[4)] #4
	\end{itemize}
	\fi
	\else
	\begin{itemize} 
		\item[1)] #1
		\item[3)] #2
		\item[2)] #3
		\item[4)] #4
	\end{itemize}
	\fi\fi
	\else
	\ifdim\wd\answera<0.4\textwidth%
	\ifdim\wd\answerb<0.4\textwidth%
	\ifdim\wd\answerc<0.4\textwidth%
	\ifdim\wd\answerd<0.4\textwidth%
	\begin{multicols}{2}
		\begin{itemize}
			\item[1)] \usebox{\answera}
			\item[3)] \usebox{\answerc}
			\item[2)] \usebox{\answerb}
			\item[4)] \usebox{\answerd}
		\end{itemize}
	\end{multicols}
	\else
	\begin{itemize} 
		\item[1)] #1
		\item[3)] #2
		\item[2)] #3
		\item[4)] #4
	\end{itemize}
	\fi
	\else
	\begin{itemize} 
		\item[1)] #1
		\item[3)] #2
		\item[2)] #3
		\item[4)] #4
	\end{itemize}
	\fi
	\else
	\begin{itemize} 
		\item[1)] #1
		\item[3)] #2
		\item[2)] #3
		\item[4)] #4
	\end{itemize}
	\fi
	\else
	\begin{itemize} 
		\item[1)] #1
		\item[3)] #2
		\item[2)] #3
		\item[4)] #4
	\end{itemize}
	\fi\fi
	\else
	\ifdim\wd\answera<0.4\textwidth%
	\ifdim\wd\answerb<0.4\textwidth%
	\ifdim\wd\answerc<0.4\textwidth%
	\ifdim\wd\answerd<0.4\textwidth%
	\begin{multicols}{2}
		\begin{itemize}
			\item[1)] \usebox{\answera}
			\item[3)] \usebox{\answerc}
			\item[2)] \usebox{\answerb}
			\item[4)] \usebox{\answerd}
		\end{itemize}
	\end{multicols}
	\else
	\begin{itemize} 
		\item[1)] #1
		\item[3)] #2
		\item[2)] #3
		\item[4)] #4
	\end{itemize}
	\fi
	\else
	\begin{itemize} 
		\item[1)] #1
		\item[3)] #2
		\item[2)] #3
		\item[4)] #4
	\end{itemize}
	\fi
	\else
	\begin{itemize} 
		\item[1)] #1
		\item[3)] #2
		\item[2)] #3
		\item[4)] #4
	\end{itemize}
	\fi
	\else
	\begin{itemize} 
		\item[1)] #1
		\item[3)] #2
		\item[2)] #3
		\item[4)] #4
	\end{itemize}
	\fi\fi
\sol}
% -------------------------------------------------------------

\begin{document}
\maketitle

\thispagestyle{empty}
\textbf{{\LARGE مقدمه مؤلفان}}
\bigskip

جزوه‌ای که هم‌اکنون زیر دست شماست، حاصل سال‌ها مطالعه و کار گروهی می‌باشد.
باشد که آدمیان زین سبب خشنود گردند.
\sol 

\begin{figure*}[bp]
\begin{center}
	\includegraphics[width=3cm]{image/QrCode/github-page}
\end{center}

\bigskip

\hfill گروه مؤلفان | شهریور سال یک‌هزار و چهارصد و یک
\end{figure*}

\newpage

\clearpage
\pagenumbering{arabic}
	
\begin{enumerate}[label={\textbf{\arabic*-}}]
	\quenstion{1}
	\quenstion{2}
	\quenstion{3}
	\quenstion{4}
	\quenstion{5}
	\quenstion{6}
	\quz{7} \bigskip \bigskip
	\quenstion{8}
	\quenstion{9}
	
	\quenstion{10}
	\quenstion{11}
	\quenstion{12}
	\quenstion{13}
	\quenstion{14}
	\quenstion{15}
	\quenstion{16}
	\quenstion{17}
	\quenstion{18}
	\quenstion{19}
	
	\quenstion{20}
	\quenstion{21}
	\quenstion{22}
	\quenstion{23}
	\quenstion{24}
	\quenstion{25}
	\quenstion{26}
	\quz{27} \bigskip \bigskip
	\quenstion{28}
	\quenstion{29}
	
	\quenstion{30}
	\quenstion{31}
	\quenstion{32}
	\quenstion{33}
	\quenstion{34}
	\quenstion{35}
	\quz{36} \bigskip \bigskip
	\quenstion{37}
	\quenstion{38}
	\quz{39} \bigskip \bigskip
	
	\quenstion{40}
	\quenstion{41}
	\quenstion{42}
	\quenstion{43}
	\quz{44} \bigskip \bigskip
	\quz{45} \bigskip \bigskip \bigskip
	\quenstion{46}
	\quenstion{47}
	\quenstion{48}
	\quenstion{49}
	
	\quenstion{50}
	\quenstion{51}
	\quenstion{52}
	\quenstion{53}
	\quenstion{54}
	\quenstion{55}
	\quenstion{56}
	\quenstion{57}
	\quz{58} \bigskip \bigskip
	\quenstion{59}
	
	\quenstion{60}
	\quenstion{61}
	\quenstion{62}
	\quenstion{63}
	\quenstion{64}
	\quenstion{65}
	\quenstion{66}
	\quenstion{67}
	\quenstion{68}
	\quenstion{69}
	
	\quz{70} \bigskip \bigskip
	\quenstion{71}
	\quenstion{72}
	\quenstion{73}
	\quenstion{74}
	\quenstion{75}
	\quenstion{76}
	\quenstion{77}
	\quenstion{78}
	\quenstion{79}
	
	\quenstion{80}
	\quz{81} \bigskip \bigskip
	\quenstion{82}
	\quz{83} \bigskip \bigskip \bigskip
	\quenstion{84}
	\quenstion{85}
	\quenstion{86}
	\quenstion{87}
	\quenstion{88}
	\quenstion{89}
	
	\quenstion{90}
	\quz{91} \bigskip \bigskip
	\quenstion{92}
	\quenstion{93}
	\quz{94} \bigskip \bigskip
	\quenstion{95}
	\quenstion{96}
	\quenstion{97}
	\quenstion{98}
	\quenstion{99}
	
	\quenstion{100}
	\quenstion{101}
	\quenstion{102}
	\quz{103} \bigskip \bigskip
	\quenstion{104}
	\quenstion{105}
	\quz{106} \bigskip \bigskip
	\quenstion{107}
	\quenstion{108}
	\quz{109} \bigskip \bigskip
	
	\quz{110} \bigskip
	\quenstion{111}
	\quenstion{112}
	\quenstion{113}
	\quenstion{114}
	\quenstion{115}
	\quenstion{116}
	\quenstion{117}
	\quenstion{118}
	\quenstion{119}
	
	\quenstion{120}
	\quenstion{121}
	\quz{122} \bigskip \bigskip
	\quenstion{123}
	\quz{124} \bigskip \bigskip
	\quenstion{125}
	\quenstion{126}
	\quenstion{127}
	\quenstion{128}
	\quenstion{129}
	
	\quenstion{130}
	\quenstion{131}
	\quenstion{132}
	\quenstion{133}
	\quenstion{134}
	\quenstion{135}
	\quz{136} \bigskip \bigskip \bigskip
	\quz{137} \bigskip \bigskip
	\quenstion{138}
	\quz{139} \bigskip \bigskip \bigskip
	
	\quenstion{140}
	\quenstion{141}
	\quenstion{142}
	\quz{143} \bigskip \bigskip
	\quz{144} \bigskip \bigskip \bigskip
	\quenstion{145}
	\quenstion{146}
	\quenstion{147}
	\quenstion{148}
	\quenstion{149}
	
	\quenstion{150}
	\quenstion{151}
	\quenstion{152}
	\quenstion{153}
	\quenstion{154}
	\quz{155} \bigskip \bigskip
	\quenstion{156}
	\quenstion{157}
	\quenstion{158}
	\quenstion{159}
	
	\quenstion{160}
	\quenstion{161}
	\quenstion{162}
	\quenstion{163}
	\quenstion{164}
	\quenstion{165}
	\quenstion{166}
	\quenstion{167}
	\quenstion{168}
	\quenstion{169}
	
	\quenstion{170}
	\quenstion{171}
	\quenstion{172}
	\quz{173} \bigskip \bigskip
	\quz{174} \bigskip \bigskip \bigskip
	\quenstion{175}
	\quenstion{176}
	\quenstion{177}
	\quenstion{178}
	\quz{179} \bigskip \bigskip \bigskip \bigskip
	
	\quz{180} \bigskip \bigskip \bigskip \bigskip
	\quz{181} \bigskip \smallskip
	\quz{182} \bigskip \smallskip
	\quenstion{183}
	\quenstion{184}
	\quenstion{185}
	\quenstion{186}
	\quenstion{187}
	\quenstion{188}
	\quz{189} \bigskip \bigskip
	
	\quenstion{190}
	\quenstion{191}
	\quenstion{192}
	\quenstion{193}
	\quz{194} \bigskip \bigskip
	\quenstion{195}
	\quenstion{196}
	\quenstion{197}
	\quenstion{198}
	\quz{199} \bigskip \bigskip \bigskip
		
	\quenstion{200}
	\quenstion{201}
	\quenstion{202}
	\quenstion{203}
	\quenstion{204}
	\quenstion{205}
	\quenstion{206}
	\quenstion{207}
	\quenstion{208}
	\quenstion{209}\newpage

\test{
به ازای چند عدد صحیح \riazi{m} نمودار تابع 
\riazi{f(x)=(2m^2+6m)x^2 -2mx -1}
از ناحیه اول \underline{نمی‌گذرد}؟
}
\answer{صفر}{۱}{۲}{۳}
\test{
در معادله درجه دوم 
\riazi{x^2 - (a-2)x + (a+1)}،
کم‌ترین مقدار مجموع مربعات ریشه‌ها کدام است؟
}
\answer{۱}{۳}{۴}{۵}
	\quz{212} \bigskip \bigskip
\test{
معادله 
\riazi{x^4 - (m^2-1)x^2+3-4m}،
چهار ریشهٔ حقیقی دارد که مجموع مربعات آن‌‌ها برابر ۳۰ است.چند مقدار برای \riazi{m} وجود دارد؟
}
\answer{صفر}{40}{۲}{۴}
	\quenstion{214} \newpage
\test{
اگر 
\riazi{\alpha} و \riazi{\beta}
ریشه‌های معادلهٔ 
\riazi{x^2 +3x -1 =0}
باشند، کدام معادلهٔ زیر ریشه‌هایش 
\riazi{\beta^2 + 3\beta^2+1}
و
\riazi{\alpha^2 + 3\alpha^2+1}
می‌باشند؟
}
\answer{$x^2+x-3=0$}{$2x^2-x+3=0$}{$x^2-3x+1=0$}{$3x^2-x-1=0$}
	\quenstion{216}
	\quenstion{217}
	\quz{218} \bigskip \bigskip \bigskip
	\quenstion{219}
	\quenstion{220}
	\quenstion{221}
	\quenstion{222}
	\quenstion{223}

	\test{
اگر 
\riazi{\alpha}
 و 
\riazi{\beta} 
اعداد طبیعی و ریشه‌های معادله 
\riazi{x^2 -(\alpha^2 + \beta^2 -12)x+\beta+\alpha-1=0}
 باشد، مقدار 
\riazi{\beta+\alpha}
  کدام است؟
	}
	\answer{2}{5}{9}{12}
	\newpage
	\test{
	به ازای دو مقدار \riazi{a}، یک ریشه معادله 
	\riazi{3x^2 -ax +4 =0}،
	سه برابر دیگر است، اختلاف این دو مقدار \riazi{a} کدام است؟
}\answer{8}{9}{16}{18}

	
	\test{
یک تاس را پرتاب می‌کنیم و عدد رو شده را به جای \riazi{m}	در معادله
\riazi{x^2 = mx +(m-1)=0}
قرار می‌دهیم. احتمال آن‌که معادله دارای دو ریشه حقیقی مثبت و متمایز باشد، کدام است؟
}\answer{$\dfrac16$}{$\dfrac13$}{$\dfrac12$}{$\dfrac23$}
	
	\quenstion{223}
	\quenstion{224}
	\quenstion{225}
	\quenstion{226}
	\quenstion{227}
	\quenstion{228}
	\quenstion{229}
	\quenstion{230}
	\quenstion{231}
	
	\test{
	اگر مجموعه جواب نامعادلهٔ
	\riazi{\dfrac{((m^2-1)x^2-4mx+4)(x-3\sqrt{x}+2)}{2x-3}\le 0}
،
به ازای 
\riazi{x>\dfrac{3}{2}}،
بازهٔ 
\riazi{[2,4]}
 باشد؛ مقدار \riazi{m}، کدام است؟
}\answer{-2}{صفر}{۱}{۲}


\quenstion{232}
\quz{233}\answer{$1<m<5$}{$2<m<5$}{$2<m<4$}{$2<m<6$}
\quz{234}\answer{$m<-6$}{$m>3$}{$0<m<3$}{$3<m<6$}\newpage
\quz{235}\answer{$-1<m<0$}{$m<0$}{$2<m<8$}{$m>8$}
\quz{236}\answer{$x^2+3x+1=0$}{$x^2-3x+1=0$}{$x^2+5x+2=0$}{$x^2-5x+2=0$}
\quz{237}\answer{$a<-4$}{$a>-4$}{$a<4$}{$a>4$}
\quz{238}\answer{$-\dfrac{9}{5}$}{$1$}{$1,-\dfrac{9}{5}$}{$-1,\dfrac{9}{5}$}
\quz{239}\answer{$-4$}{$4,-12$}{$-4,12$}{$12$}\newpage

\quz{240}\answer{$-4$}{$-2$}{$2$}{$4$}
\quz{241}\answer{$-2$}{$-1$}{$1$}{$2$}
\quz{242}\answer{$-1<m<5$}{$-1<m<4$}{$-2<m<4$}{$-1<m<5$}
\quz{243}\answer{$-\dfrac{3}{2}<m<5$}{$0<m<2$}{$\dfrac{3}{2}<m<\dfrac{5}{2}$}{$\dfrac{3}{2}<m<2$}
\quz{244}\answer{$m\ge1$}{$m<2$}{$1\le m<2$}{هیچ مقدار $m$}\newpage
\quz{245}\answer{$-2$}{$-1$}{$\dfrac{2}{3}$}{$\dfrac{4}{2}$}
\quz{246}\answer{$-14$}{$-12$}{$-8$}{$-6$}
\quenstion{247}
\quenstion{248}
\quenstion{249}

\quenstion{250}
\quenstion{251}
\quenstion{252}
\quenstion{253}
\quenstion{254}
\quenstion{255}
\quenstion{256}
\quenstion{257}
\quenstion{258}
\quenstion{259}

\quenstion{260}
\quenstion{261}
\quenstion{262}
\quenstion{263}
\quenstion{264}
\quenstion{265}
\quenstion{266}
\quenstion{267}
\quenstion{268}
\quenstion{269}

\quenstion{270}
\quenstion{271}
\quenstion{272}
\quenstion{273}
\quenstion{274}
\quenstion{275}
\quenstion{276}
\quenstion{277}
\quz{278} \bigskip \bigskip
\quenstion{279}

\quenstion{280}
\quenstion{281}
\quenstion{282}
\quenstion{283}
\quenstion{284}
\quenstion{285}
\quenstion{286}
\quenstion{287}
\quenstion{288}
\quenstion{289}

\quenstion{290}
\quenstion{291}
\quenstion{292}\newpage
\quz{293}\answer{$m<0$}{$m>4$}{$-1<m<4$}{$-2<m<6$}
\quz{294}\answer{$-1$}{2}{4}{6}
\quz{295}\answer{$3.5$}{4}{$4.5$}{5}
\quz{296}\answer{2}{3}{4}{6}
\quz{297}\answer{$(-\infty,2) \cup (6,+\infty)$}{$(-\infty,3) \cup (4,+\infty)$}{$(2,6)$}{$(3,4)$}\newpage
\quz{298}\answer{1}{2}{3}{4}
\quz{299}\answer{$-1$}{$-10$}{8}{6}

\quz{300}\answer{$-6,-2$}{$-6,2$}{$6,-1$}{$-6,1$}
\quz{301}\answer{$-4$}{$-3$}{3}{4}
\quz{303}\answer{$m<-4$}{$m>4$}{$-4<m<4$}{$4<m<9$}

\newpage

\test{
معادلهٔ
\riazi{\dfrac{1}{\sqrt{2-x}+2}- \dfrac{1}{2- \sqrt{2-x}} = \dfrac{2-x}{5\sqrt{2-x}}}
چند ریشه دارد؟
}\answer{صفر}{1}{2}{3}

\test{
اگر سهمی
\riazi{y=-ax^2+ax+2}
روی سهمی
\riazi{y=2bx^2-bx-1}
قرار داشته باشد و برعکس، مقدار
\riazi{a-b}
چقدر است؟
}\answer{$-6$}{$6$}{$-18$}{18}

\test{
اگر
\riazi{\alpha} و \riazi{\beta}
ریشه‌های معادلهٔ 
\riazi{4x^3+kx^2-9x-2=0}، \riazi{\alpha + \beta = 1} و \riazi{\alpha\beta = -2}
باشد، مقدار 
\riazi{k} چقدر است؟
}\answer{$-\dfrac{27}{5}$}{$\dfrac{27}{5}$}{$-3$}{3}


\test{
فاصله نقطهٔ تلاقی منحنی‌های
\riazi{2y=x^2} و \riazi{x=\sqrt{y+3}-\sqrt{y-3}}
با مبدأ مختصات، کدام است؟
}\answer{$\sqrt{3}$}{$\sqrt{6}$}{$2\sqrt{3}$}{$\sqrt{15}$}

\test{
معادلهٔ
\riazi{\dfrac{\sqrt{x+1}}{\sqrt{x-1}+3} - \dfrac{\sqrt{x+1}}{3-\sqrt{x-1}} = \dfrac{x-1}{\sqrt{x-1}}}
چند ریشهٔ مثبت دارد؟
}\answer{صفر}{1}{2}{3}
\newpage
\test{
به ازای چند مقدار
\riazi{y=ax^2+(3+2a)x}
از ناحیه سوم محور‌های مختصاتی \underline{نمی‌گذرد}؟
}\answer{هیچ مقدار $a$}{تمام مقادیر $a$}{1}{2}
\end{enumerate}

\end{document}